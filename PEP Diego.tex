\documentclass[12pt]{article}
\usepackage{plain}
\usepackage{float}
\usepackage[utf8]{inputenc} 
\usepackage{graphicx} 
\usepackage{tabularx}
\usepackage[brazil]{babel}


\begin{document}
	
\begin{titlepage}
	\thispagestyle{empty}
	
	\vfill 
	
	
	\begin{center}
		
		\resizebox{!}{0.4cm}{UNIVERSIDADE FEDERAL DO RIO GRANDE DO SUL}\\
		
		
		\resizebox{!}{0.5cm}{INSTITUTO DE INFORMÁTICA} \\
		\resizebox{!}{0.5cm}{PROGRAMA DE PÓS-GRADUAÇÃO EM COMPUTAÇÃO}
		
		\vspace*{2.5cm}
		
		
		\Large{{\bf Uma Abordagem para a Verificação de Regras de Modelagem para Modelos de Processo de Negócio }}
		
		\vspace*{2cm}
		
		%Autor
		
		\Large{Diego Toralles Avila\\ \resizebox{!}{0.3cm}{\textit{dtavila@inf.ufrgs.br}}}\\ 
		% \Large{Nome do segundo autor} % se houver outro autor, coloque aqui.
		
		
		\vspace*{1cm}
		
		%Orientador
		\normalsize{Orientadora: Dra. Lucinéia Heloisa Thom}
		
		\vspace*{1.5cm}
		\begin{flushright} 
			\parbox[l]{9.3cm}{
				
				\normalsize{\emph{Plano de Estudos e Pesquisa}}}
		\end{flushright}
		\vspace*{2cm}
		
		\large{Porto Alegre, Outubro de 2016}
	\end{center}
\end{titlepage}

\section{Introdução}

%Introduzir BPM
Gerenciamento de Processos de Negócio (Business Process Management - BPM) é uma disciplina que provêm uma abordagem sistemática para gerenciar o trabalho de uma organização por meio da modelagem, da análise, da melhoria e do controle de seus processos. Ela permite o aumento da produtividade e a redução dos custos de trabalho através de processos mais eficazes, mais eficientes e mais adaptáveis \cite{aalst:2013}. Desta forma, organizações estão cada vez mais preocupados com a qualidade dos seus  processos e também, por causa disto, com a qualidade da sua modelagem,  já que modelos de alta qualidade influenciam a qualidade do próprio processo.

%Introduzir Modelagem
Modelagem não é uma tarefa fácil nem objetiva. Apesar das ferramentas de modelagem de processos ajudarem seus usuários, experientes ou não, a criar modelos de processo, elas não podem garantir a validade nem a compreensão destes modelos, pois muito da dificuldade da modelagem esta em descobrir o que o processo de fato faz e como representar, dentro do modelo, cada descoberta de forma precisa e facilmente compreensível. Logo, muito do esforço dedicado a modelagem depende das pessoas que estão criando o modelo, sendo difícil garantir a sua qualidade \cite{Mendling2008}

%Introduzir Qualidade
Esta dificuldade impulsionou o desenvolvimento de diferentes abordagens sobre como considerar a qualidade de um modelo de processo. O \textit{Framework} SEQUAL \cite{Krogstie2006,Lindland1994} é uma destas abordagens, construído com base na teoria de semiótica e que define diversos aspectos de qualidade baseados no relacionamentos entre um modelo, um conjunto de conhecimentos, um domínio, uma linguagem de modelagem e as atividades de aprendizado, de tomada de ações e de modelagem \cite{Mendling2007}. De acordo com este \textit{Framework}, a qualidade pode ser dividida em sete partes:

\begin{description}
	\item [Qualidade Física] - A persistência, a vigência e a disponibilidade de um modelo.
	\item [Qualidade Empírica] - O relacionamento entre o modelo e outro modelo contendo as mesmas afirmações que é considerado, de alguma forma, melhor através de uma diferente organização ou layout.
	\item [Qualidade Sintática] - O relacionamento entre o modelo e a linguagem utilizada para a modelagem. A linguagem foi utilizada corretamente?
	\item [Qualidade Semântica] - O relacionamento entre o modelo e o domínio da modelagem. O modelo esta completo (contêm todas as afirmações válidas) e é válido (não contêm uma afirmação inválida):
	\item [Qualidade Pragmática] - O relacionamento entre o modelo e a interpretação das partes interessadas pelo o modelo. A audiência entende as implicações das partes do modelo relevantes a eles?
	\item [Qualidade Social] - O relacionamento entre as diferentes interpretações do modelo. Os diferentes participantes da modelagem concordam sobre a qualidade semântica do modelo?
	\item [Qualidade Deôntica] - Como o modelo contribui para atingir os objetivos gerais da modelagem?
\end{description}

A qualidade pragmática é um importante objetivo de um modelo de processo. Nem mesmo o melhor modelo possível será útil se ele não for entendido, seja por uma pessoa ou por uma máquina. Logo, qualquer interpretação do modelo deve estar correta em relação ao que foi pretendido expressar, pois deve ser possível acompanhar o comportamento real do processo ao realizar uma encenação ou a execução deste \cite{Krogstie2012}.

Para um modelo conseguir a atingir a sua qualidade pragmática ele é preciso ser compreensível. Um revés disto é a inconveniência de testar a compreensão de um modelo, pois este só pode ser compreensível se ele for avaliado e entendido por um interpretador. Entretanto, a compreensibilidade do modelo, que esta sobre o domínio da qualidade empírica, obviamente influencia o quão compreensível é um modelo. De fato, a qualidade empírica tende a beneficiar a qualidade pragmática \cite{Krogstie2012}.
%Compreensibilidade, ao invés da compreensão, esta sobre o domínio da qualidade empírica de um modelo. De fato, a qualidade empírica tende a beneficiar a qualidade pragmática \cite{Krogstie2012}.

Garantir a compreensão humana de um modelo de processo também não é fácil. Um modelo de processo pode apresentar obstáculos a compreensão por causa da complexidade do modelo ou da infamiliaridade com a linguagem de modelagem utilizada. Existem, entretanto, maneiras de como atingir a qualidade pragmática, uma delas sendo a transformação do modelo para outro com maior compreensibilidade \cite{Krogstie2012}. 

%Introduzir Guidelines
Esta transformação pode ser feita pelo a aplicação de regras de modelagem, cujo propósito é restringir a introdução de construções indesejadas ao modelo do processo, para, desta forma, ajudar o modelador a reduzir a complexidade e o número de erros em um modelo de processo. Estas regras de modelagem são convenções criadas para descrever quais características (como o número de elementos ou o estilo da rotulação) um modelo de processo deve possuir para aumentar a compreensibilidade deste.

%Introduzir o trabalho
Existem diversos trabalhos na literatura propondo regras de modelagem de ambos os praticantes de BPM \cite{Silver2009} \cite{White2008} \cite{Allweyer2010} quanto pesquisadores \cite{Becker2000} \cite{Mendling2007} \cite{Vanderfeesten2008} \cite{Correia2012}. A tabela \ref{7PMG} lista, como exemplo, as regras descritas por Mendling. Estes trabalhos sumarizam em suas propostas o conhecimento e/ou a pesquisa dos autores sobre quais tipos de modelos de processo possuem melhor compreensibilidade. Contudo, é difícil extrair de toda a literatura quais regras de modelagem possuem validade empírica, principalmente porque as regras propostas por diferentes autores muitas vezes não são precisas, possuem ambiguidades ou diretamente contradizem umas as outras. 

	\begin{table}
		\begin{tabularx}{\textwidth}{l X}\label{7PMG}
			Número & Régra\\\hline
			G1 & \textbf{Nodes} - Do not use more than 31.\\
			G2 & \textbf{Conn. Degree} - No more than 3 inputs or outputs per connector.\\
			G3 & \textbf{Start and End} - Use no more than 2 start and end events.\\
			G4.a & \textbf{Structuredness} - Model as structured as possible.\\
			G4.b & \textbf{Mismatch} - Use design patterns to avoid mismatch.\\
			G5.a & \textbf{OR-connectors} - Avoid OR-joins and OR-splits.\\
			G5.b & \textbf{Heterogeneity} - Minimize the heterogeneity of connector types.\\
			G5.c & \textbf{Token Split} - Minimize the level of concurrency.\\
			G6 & \textbf{Text} - Use verb-object activity labels.\\
			G7 & \textbf{Nodes} - Decompose a model with more than 31 nodes\\
		\end{tabularx}
		\caption{Regras de modelagem propostas por Mendling \cite{Mendling2013}}
	\end{table}


Além disto, a aplicabilidade das regras de modelagem como um recurso proativo para criar modelos de processo com maior compreensibilidade ainda apresenta desafios. Recorrentemente, o uso destas regras na literatura limita-se a análise do modelo de processo já completado, ou seja, a análise retroativa de um modelo de processo com todos os elementos já inseridos. Algumas ferramentas de modelagem, como a Signavio ou a Activiti, apresentam recursos para dar suporte a criação de modelos com maior qualidade pragmática, mas em grande parte estes recursos são limitados e não foram criados com base em regras de modelagem.

Em vista disto, o trabalho proposto aqui procura saber quais regras de modelagem válidas existem na literatura e criar uma abordagem que aplica proativamente estas regras para o desenvolvimento de modelos de processo que tenham maior compreensibilidade.

\section{Objetivos}

\subsection{Objetivo Geral}
%Pesquisar a aplicabilidade proativa das regras de modelagem propostas na literatura para a criação de modelos de processos de negócio com maior qualidade pragmática.% e comparar com o suporte dado a estas pelas ferramentas de modelagem em BPMN atuais.

Desenvolver uma abordagem para a verificação proativa de modelos de processo de negócio utilizando as regras de modelagem propostas na literatura.

\subsection{Objetivos Específicos}
\begin{itemize}
	\item Descobrir quais regras de modelagem de processos existem na literatura.
	\item Verificar quais regras de modelagem foram validadas de forma empírica.
	\item Analisar qual o suporte dado por ferramentas de modelagem existentes para a verificação de regras de modelagem.
	\item Comparar o suporte dado pelas ferramentas analisadas entre si e com as regras de modelagem descobertas.
	\item Estender uma ferramenta para verificar as regras de modelagem descobertas que possuem validade empírica.
\end{itemize}

\section{Resultados Esperados}

Como contribuição principal deste trabalho, espera-se que os resultados da pesquisa possam auxiliar no desenvolvimento de métodos e ferramentas que dão maior suporte para a criação de modelos de processos com maior compreensibilidade. Desta forma, os resultados esperados são:

\begin{itemize}
	\item A seleção de um conjunto de regras de modelagem extraídas da literatura, classificadas conforme a sua validade empírica.
	\item A análise de ferramentas de modelagem sobre o suporte a verificação de regras de modelagem.
	\item Uma abordagem para a verificação de regras de modelagem em uma ferramenta.
\end{itemize}

\section{Metodologia}

A metodologia deste trabalho baseia-se em um estudo inicial realizado sobre qualidade de modelos de processo de negócio. Neste estudo, foi identificado o problema de pesquisa já descrito aqui. Para cumprir os objetivos da pesquisa, é necessário realizar os próximos passos:
\begin{enumerate}
	\item Identificar as regras de modelagem existentes na literatura.
	\item Classificar e selecionar as regras de modelagem de acordo com a sua validade empírica.
	\item Analisar o suporte dado pelas ferramentas de modelagem às regras de modelagem extraídas.
	\item Analisar os resultados obtidos, objetivando criar a abordagem para uma aplicação das regras de modelagem selecionadas. 
\end{enumerate}

Na primeira etapa, será realizado uma revisão sistemática em busca de trabalhos que descrevem algum conhecimento sobre modelos de processo com maior compreensibilidade. 

%Ferramentas, boas publicacoes, acad..
%Verifica ambiguidades
Durante esta revisão sistemática, também deve ser extraída informações sobre a validade empírica do conhecimento descrito nos trabalhos. Estas informações serão utilizadas na segunda etapa, para selecionar quais regras de modelagem são mais relevantes para a aplicação na nossa abordagem. Para esta seleção leva-se em conta em que publicações estas regras foram propostas, se há contradições entre múltiplas regras extraídas e se foram realizados experimentos que tentam provar a utilidade destas regras no aumento da compreensibilidade de um modelo de processo.

Na terceira etapa, a análise das ferramentas de modelagem existentes procura verificar o suporte destas ferramentas para criar modelos de processo com maior compreensibilidade. A partir disto, é possível analisar quais regras de modelagem que são suportadas pelas ferramentas e as maneiras que estas regras foram verificadas.

Por fim, a ultima etapa estuda os resultados obtidos. Será discutido quais regras de modelagem são suportadas pelas ferramentas de modelagem e quais os tipos de suporte. Também irá ser verificado qual o tipo mais comum de regras que são suportadas, além de apontar quais precisam de mais suporte no desenvolvimento de ferramentas futuras. Isto tudo será unificado em uma abordagem de verificação automática das regras de modelagem.

Não se descarta a possibilidade de que seja necessário, em um trabalho futuro, estender uma ferramenta de modelagem para estudar o efeito das regras de modelagem que não forem contempladas na terceira etapa.

\section{Cronograma}

Para realização da pesquisa proposta nesta dissertação, foi elaborado um cronograma que contempla as atividades necessárias para seu desenvolvimento. Algumas destas atividades já foram realizadas, como a pesquisa sobre as abordagens existentes sobre qualidade de modelos de processo. O cronograma define as próximas atividades a serem realizadas, em ordem cronológica.

\begin{enumerate}
	\item Revisão sistemática para a extração de regras de modelagem para o aumento da qualidade pragmática de um modelo.
	\item Análise da validade empírica das regras extraídas.
	\item Seleção de ferramentas de modelagem para a análise.
	\item Classificação do suporte a regras de modelagem de cada ferramenta selecionada.
	\item Comparação das ferramentas selecionadas.
	\item Estudo dos resultados.
	\item Criação da Abordagem
	\item Escrita da dissertação.
	\item Entrega da dissertação.
\end{enumerate}

\newpage
\section{Referências}
\bibliographystyle{plain}
\bibliography{ArtigoVerificacao2}

\end{document}